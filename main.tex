\typeout{ ====================================================================}
\typeout{ this is file main.tex, created at 02-Jun-2016               }
\typeout{ maintained by Gustavo Rabello dos Anjos                             }
\typeout{ e-mail: gustavo.rabello@gmail.com                                   }
\typeout{ ====================================================================}

\typeout{ ====================================================================}
\typeout{ this is file preamble.tex, created at 02-Jun-2016               }
\typeout{ maintained by Gustavo Rabello dos Anjos                             }
\typeout{ e-mail: gustavo.rabello@gmail.com                                   }
\typeout{ ====================================================================}

\documentclass[12pt,a4paper,fleqn,portuguese]{report}
\usepackage{babel,subfig}
\usepackage[utf8]{inputenc}
\usepackage[top=1.5cm,bottom=1.5cm,left=1.5cm,right=1.5cm]{geometry}
\usepackage{graphicx}    % pacote para inclusao de figuras,
\newcommand{\HRule}{\rule{\linewidth}{0.5mm}}

\typeout{ ****************** End of file preamble.tex ****************** }



\begin{document}	
% no page numbering
\pagenumbering{gobble}

\typeout{ ====================================================================}
\typeout{ this is file header.tex, created at 02-Jun-2016                     }
\typeout{ maintained by Gustavo Rabello dos Anjos                             }
\typeout{ e-mail: gustavo.rabello@gmail.com                                   }
\typeout{ ====================================================================}


\begin{center}

\begin{figure}[ht!]
	\captionsetup[subfigure]{labelformat=empty}
	\begin{center}
		\subfloat[]{
			\includegraphics[scale=0.2]{./figs/uerj.png} }
		\qquad
		\subfloat[]{
		 	\includegraphics[scale=0.6]{./figs/ppgem.png} }
	\end{center}
\end{figure}
\vspace{-1.0cm}

\textsc{Universidade do Estado do Rio de Janeiro - UERJ}\\[0.1cm]
\textsc{Programa de Pós-Graduação em Engenharia Mecânica - PPG-EM}\\[0.1cm]
\textsc{4\textsuperscript{\b o} Seminário PPG-EM UERJ 2016}\\[0.1cm]
% Bottom of the page
{\today}
\HRule

\textsc{Short Bio}

\vspace{0.5cm}

\end{center}

\typeout{ ****************** End of file header.tex ****************** }



\Large

\noindent \textbf{Professor Marcelo Savi} é graduado em Engenharia
Mecânica e obteve o título de doutor em Engenharia Mecânica na  PUC-Rio
em 1994. Atualmente é Professor Titular do Programa de Engenharia
Mecânica - COPPE/UFRJ, pesquisador 1A do CNPq, Cientista do Nosso Estado
(FAPERJ) e também coordenador do Centro de Mecânica Não-Linear. Possui
mais de 350 publicações em congressos e revistas internacionais e já
orientou e ainda orienta diversos trabalhos de fim de curso, mestrado e
doutorado, totalizando mais de 100. É membro da ABCM, onde participa dos
comitês de Dinâmica; Materiais e Estruturas Inteligentes e Fenômenos
Não-lineares e Caóticos. Tem interesse em mecânica não-linear onde
destacam-se materiais e estruturas inteligentes; dinâmica não-linear,
caos e controle; biomecânica e meio ambiente.

\vspace{1cm}

\noindent \textbf{Dr. Vitor Ramos} é graduado em física e obteve o título de
doutor em ENGENHARIA NUCLEAR com ênfase em FÍSICA NUCLEAR APLICADA pela
UFRJ em 2010. Foi pesquisador do CBPF. Já desenvolveu trabalhos no
Instituto de Engenharia Nuclear IEN/CNEN, sendo especialista em
radiotraçadores com trabalhos voltados à calibração e aferição de
medidores de vazão, medidas e modelagem de fluxos líquidos/gasosos e
desenvolvimento de filtros especiais para imobilização de gases
radioativos. Atualmente desenvolve projetos voltados ao novo laboratório
de nanotecnologia do PPG-EM. Suas linhas de pesquisa são relacionadas à
análise em textura e stress em placas metálicas, operação de
difratômetros, análise de filmes finos, conservação e manutenção em
equipamentos de raio-x.



\end{document}

\typeout{ ****************** End of file main.tex ****************** }

